\documentclass[]{article}
\usepackage{graphicx}
\usepackage{hyperref}
%opening
\title{}
\author{}

\begin{document}
\begin{center}
	Laboratorio Introducción a Packet Tracer \\
	Estefanía Leva Ungaro \\
	August 29, 2024
\end{center}
\maketitle


\section{Packet Tracer}
Packet Tracer es una herramienta para aprender y practicar sobre las configuraciones de una red; y a la vez,los dispositivos y cables que están involucrados en la misma. Dentro de ésta aplicación podemos encontrar un modo "Lógico"(Logical) dónde podemos ver las conexiones de forma ilustrada y,por otro lado tenemos el modo "Fisico"(Physical) donde muestra los dispositivos y cableados de forma más real (Figura 1). Tenemos también el modo "Realtime" que muestra las comunicaciones en el momento y el modo "Simulation" en el que podemos controlar la velocidad en la que se realiza la comunicación entre dispositivos.(Figura 2)
\begin{center}
	\includegraphics[width=0.5\textwidth]{modos-log-phy.png} \\  % Imagen centrada
	Figura 1
\end{center}

\vspace{10pt}  % Espacio vertical entre las imágenes

\begin{center}
	\includegraphics[width=0.5\textwidth]{modos-simul.png} \\  % Imagen centrada
	Figura 2
\end{center}
\section{Dispositivos finales}
En los dispositivos finales tenemos:PC,Laptop,Server e impresora con conectividad de red,entre otros.
\begin{center}
	\includegraphics[width=0.5\textwidth]{Dispositivos-fin.png} \\  % Imagen centrada
	Figura 3
\end{center}
\section{Dispositivos de red}
Dentro de los dispositivos de red,contamos con:
Routers:son el punto de conexión entre una red local e Internet.
\begin{center}
	\includegraphics[width=0.5\textwidth]{router.png} \\  % Imagen centrada
	Figura 4
\end{center}
Switch: permiten la conexión entre distintos dispositivos dentro de una red.
\begin{center}
	\includegraphics[width=0.5\textwidth]{switch.png} \\  % Imagen centrada
	Figura 5
\end{center}
Hub: permite la conexión de muchos dispositivos dentro de una red,actuan también como repetidores y actualmente en su mayoría son reemplazados por switches.
\begin{center}
	\includegraphics[width=0.5\textwidth]{hub.png} \\  % Imagen centrada
	Figura 6
\end{center}
\section{Cableado}
Dentro de los cableados que nos ofrece Packet Tracer tenemos: USC,Coaxial,automático,fibra,entre otros.
\begin{center}
	\includegraphics[width=0.5\textwidth]{cableado.png} \\  % Imagen centrada
	Figura 7
\end{center}
\section{Conectar 2 PC}
Para conectar 2 PC requeriremos arrastras 2 PC's desde los "Dispositivos Finales" y conectar mediante un cableado automático.
\begin{center}
	\includegraphics[width=0.5\textwidth]{2pc.png} \\  % Imagen centrada
	Figura 8
\end{center}
\section{Referencias}

Curso Introductorio Cisco Packet Tracer de Netacad: \href{https://www.netacad.com/launch?id=ec0847b7-e6fc-4597-bc31-38ddd6b07a2f&tab=curriculum&view=2e838579-e095-5f66-bb62-c440769a0b24}{Netacad - Curso Introductorio Cisco Packet Tracer}

Mi repositorio: \href{https://github.com/estefilungaro/Laboratorio_packet_tracer}{GitHub - Laboratorio Packet Tracer}
\end{document}
